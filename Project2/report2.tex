\documentclass{article}
\usepackage[utf8]{inputenc}
\usepackage{graphics}
\usepackage{indentfirst}

\title{VE281 Lab2 Report}
\author{Haoyun Zhou}
\date{\today}

\begin{document}
\maketitle

\section{Introduction}
In this report, the performance of HashTable implemented in the lab, unordered\_map in STL, and list in STL are compared. Same random data is generated for each data structure as the following way. Denote the number of operations by $n$. Let keys be integers and range from 1 to $m$. Each operation can be insertion, deletion, and accession in random.

\section{Results}
When $n=1000000, m=1000$, HashTable, unordered\_map, and list spends 0.127477s, 0.0958371s, and 4.73161s respectively. \par
When $n=100000, m=1000$, HashTable, unordered\_map, and list spends 0.0128269s, 0.0098627s, and 0.470215s respectively. \par
When $n=100000, m=10000$, HashTable, unordered\_map, and list spends 0.0152615s, 0.0107248s, and 4.50088s respectively.

\section{Discussion}
In all cases above, unordered\_map spends the shortest time, while list spends the longest time. Time spent by HashTable is close to that of unordered\_map. \par
When $n=100000$ is fixed and $m$ changes, time consumed by HashTable and unordered\_map is in the same order of magnitude, while time consumed by list increases by an order of magnitude as $m$ increases by an order of magnitude. It is because the time complexity of a single operation for HashTable and unordered\_map is $O(1)$, while that for list is $O(m)$. \par
When $m=1000$ is fixed and $n$ changes from 1000 to 10000, the time consumed by all data structures increase an order of magnitude. \par
In conclusion, the time complexity of HashTable and unordered\_map is $O(n)$, while that for list is $O(nm)$.

\end{document}